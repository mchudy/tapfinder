\documentclass[a4paper]{article}

\usepackage[T1]{fontenc}
\usepackage{polski}
\usepackage[utf8]{inputenc}
\usepackage[a4paper,left=2.5cm, right=2.5cm, top=2.5cm, bottom=2.5cm]{geometry}
\usepackage[polish]{babel}
\usepackage{hyperref}
\hypersetup{
    colorlinks=true,
    linkcolor=blue,
    urlcolor=blue
}
\usepackage{amsmath}

\title{Android: Raport końcowy}
\author{Marcin Chudy, Grzegorz Czarnocki}

\begin{document}
\maketitle

\section{Co powinno znaleźć się w tym raporcie}

Końcowy raport powinien zawierać informacje dotyczące wykorzystanych narzędzi (środowisko, biblioteki itp., w szczególności nietypowe rozwiązania), podział pracy (bardzo istotna część raportu), krótki opis aplikacji (bez screenów), motywację oraz potencjał komercyjny aplikacji (max. 5 stron na wszystko, dowolny layout).

\section{Wstęp}

TapFinder to aplikacja, która służy do wyszukiwania miejsc, w których można napić się dobrego piwa w okolicy. Pozwala ona na rejestrację użytkownika (standardową, oraz przez Facebooka), wyszukanie piw z danej kategorii, przeglądanie miejsc na mapie oraz ich szczegółów.

\section{Technologie}

\subsection{Aplikacja mobilna}
Aplikacja mobilna wspiera urządzenia z systemem Android w wersji co najmniej 4.4 KitKat (API Level 19). Została napisana z wykorzystaniem narzędzia Android Studio.

\subsubsection{Użyte biblioteki}
\begin{itemize}
\item ButterKnife - wstrzykiwanie widoków do aktywności i fragmentów, skutkujące czystszym kodem,
\item Dagger - kontener Dependency Injection,
\item EventBus - mechanizm publikowania/subskrybowania zdarzeń, ułatwiający komunikację pomiędzy komponentami,
\item Retrofit - obsługa zapytań HTTP,
\item RxJava i RxAndroid - obsługa asynchronicznych operacji (głównie zapytań HTTP),
\item Timber - logowanie,
\item Lombok - użycie adnotacji do automatycznej generacji kodu,
\item Retrolambda - dostępność funkcji lambda obecnych w Javie 8,
\item Picasso - ładowanie i przetwarzanie obrazów,
\end{itemize}

\subsection{Aplikacja serwerowa}
Aplikacja serwerowa została napisana w języku C\# oraz technologii ASP.NET Web API 2 z użyciem środowiska Microsoft Visual Studio 2015. Wykorzystana baza danych to Microsoft SQL Server 2014. Serwer został uruchomiony z wykorzystaniem chmury \href{https://aws.amazon.com/?nc2=h_lg}{Amazon Web Services} i jest obecnie dostępny pod adresem \url{https://tapfinderapp.tk}. 

\subsubsection{Użyte biblioteki}
\begin{itemize}
\item Autofac - kontener Dependency Injection
\item Entity Framework - ORM
\item AutoMapper - mapowanie encji na obiekty DTO i odwrotnie
\item DbUp - migracje SQL
\item NLog - logowanie
\item FAKE - system budowania projektu
\end{itemize}

\section{Podział pracy}

Podział pracy w projekcie nie był jednolity. 

\section{Motywacja}

Pierwszą zamysł dotyczący aplikacji był związany z pomysłem dotyczącym restauracji. Aplikacja miałaby za zadanie zrzeszać pod jednym szyldem informacje na temat restauracji, w początkowej fazie pomysłu - z obszaru Warszawy. Szybko jednak zmieniono ten pomysł na bardziej ukierukowany, i tym kierunkiem okazały się piwa. 

\section{Potencjał komercyjny aplikacji}

Aplikacja TapFinder ma potencjał komercyjny.

\end{document}