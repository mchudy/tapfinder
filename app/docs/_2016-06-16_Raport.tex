\documentclass[a4paper]{article}

\usepackage[T1]{fontenc}
\usepackage{polski}
\usepackage[utf8]{inputenc}
\usepackage[a4paper,left=2.5cm, right=2.5cm, top=2.5cm, bottom=2.5cm]{geometry}
\usepackage[polish]{babel}
\usepackage{hyperref}
\hypersetup{
    colorlinks=true,
    linkcolor=blue,
    urlcolor=blue
}
\usepackage{amsmath}

\title{Android: Raport końcowy}
\author{Marcin Chudy, Grzegorz Czarnocki}

\begin{document}
\maketitle

\section{Co powinno znaleźć się w tym raporcie}

Końcowy raport powinien zawierać informacje dotyczące wykorzystanych narzędzi (środowisko, biblioteki itp., w szczególności nietypowe rozwiązania), podział pracy (bardzo istotna część raportu), krótki opis aplikacji (bez screenów), motywację oraz potencjał komercyjny aplikacji (max. 5 stron na wszystko, dowolny layout).

\section{Wstęp}

\textit{TapFinder} to aplikacja, która służy do wyszukiwania miejsc, w których można napić się dobrego piwa w okolicy. Pozwala ona na rejestrację użytkownika (standardową, oraz przez Facebooka), wyszukanie piw z danej kategorii, przeglądanie miejsc na mapie oraz ich szczegółów.

\section{Technologie}

\subsection{Aplikacja mobilna}
Aplikacja mobilna wspiera urządzenia z systemem Android w wersji co najmniej 4.4 KitKat (API Level 19). Została napisana z wykorzystaniem narzędzia Android Studio.

\subsubsection{Użyte biblioteki}
\begin{itemize}
\item ButterKnife - wstrzykiwanie widoków do aktywności i fragmentów, skutkujące czystszym kodem,
\item Dagger - kontener Dependency Injection, pozwala na łatwe wstrzykiwanie zależności,
\item EventBus - mechanizm publikowania / subskrybowania zdarzeń, ułatwiający komunikację pomiędzy komponentami w kod, pozwala na łączenie aktywności, fragmentów w całość,
\item Retrofit - obsługa zapytań HTTP, służy zamianie HTTP API na interfejs Javy,
\item RxJava i RxAndroid - obsługa asynchronicznych operacji (głównie zapytań HTTP),
\item Timber - logowanie, rozszerzenie zwykłej klasy Log z Androida,
\item Lombok - użycie adnotacji do automatycznej generacji kodu, konstruktorów, publicznych właściwości itp. 
\item Retrolambda - dostępność funkcji lambda obecnych w Javie 8,
\item Picasso - ładowanie i przetwarzanie obrazów,
\end{itemize}

\subsection{Aplikacja serwerowa}
Aplikacja serwerowa została napisana w języku C\# oraz technologii ASP.NET Web API 2 z użyciem środowiska Microsoft Visual Studio 2015. Wykorzystana baza danych to Microsoft SQL Server 2014. Serwer został uruchomiony z wykorzystaniem chmury \href{https://aws.amazon.com/?nc2=h_lg}{Amazon Web Services} i jest obecnie dostępny pod adresem \url{https://tapfinderapp.tk}. 

\subsubsection{Użyte biblioteki}
\begin{itemize}
\item Autofac - kontener Dependency Injection, pozwala na wstrzykiwanie zależności w kod, 
\item Entity Framework - ORM, mapowanie obiektowo-relacyjne,
\item AutoMapper - mapowanie encji na obiekty DTO i odwrotnie,
\item DbUp - migracje SQL, 
\item NLog - logowanie
\item FAKE - system budowania projektu
\end{itemize}

\section{Podział pracy}

Podział pracy w projekcie nie był jednolity. Marcin Chudy zajął się zarówno aplikacją mobilną, jak i serwerem. Z racji większego doświadczenia w developowaniu tego typu aplikacji, to ze strony Marcina Chudego powstał serwer, uruchomiony za pomocą AWS. W przypadku aplikacji mobilnej, również postarał się o przygotowanie środowiska pod jej rozwijanie.

Grzegorz Czarnocki był odpowiedzialny za stronę back-endu (rozwijanie bazy danych, migracje, kontrolery), oraz po części za stronę mobilną. Z racji posiadania komputera, który nie był dostosowany do komfortowej pracy z narzędziem Android Studio, wkład jest mniejszy, niż w przypadku p. Marcina, jednak przez okres trwania projektu na bieżąco śledzone były wszystkie zmiany, które odbywały się w projekcie.

\section{Motywacja}

Pierwszą zamysł dotyczący aplikacji był związany z pomysłem dotyczącym restauracji. Aplikacja miałaby za zadanie zrzeszać pod jednym szyldem informacje na temat restauracji, w początkowej fazie pomysłu - z obszaru Warszawy. Szybko jednak zmieniono ten pomysł na bardziej ukierukowany, i tym kierunkiem okazały się piwa. Motywacja, która stoi za aplikacją \textit{TapFinder} jest następująca: zapewnić aplikację, która w łatwy sposób dałaby możliwość użytkownikom na to, by znaleźć promocje na trunki w ich ulubionych browarniach i pubach. I to jest osiągalne z użyciem tej aplikacji.

\section{Potencjał komercyjny aplikacji}

Aplikacja \textit{TapFinder} ma potencjał komercyjny. Związany jest on z ciekawą tematyką, która poruszona jest przez aplikację, czyli po prostu... piwa. Uwagę przykuwa estetyczny, ładny interfejs graficzny, który nie jest rozbudowany, a prosty w obsłudze i intuicyjny. Mapa jako główny element aplikacji jest oparta na technologii Google Maps, znanej niemalże każdemu. Logowanie może odbywać się zarówno przez e-mail, hasło, jak i przez Facebooka. Zdjęcia w tle każdego miejsca pobierane są automatycznie z Google Places API.

W kilku słowach, jest to aplikacja, która przykuwa wzrok. I to, w połączeniu z prostotą interfejsu, jest zaletą aplikacji \textit{TapFinder}.

\end{document}